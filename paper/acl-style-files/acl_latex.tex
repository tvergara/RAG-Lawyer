% This must be in the first 5 lines to tell arXiv to use pdfLaTeX, which is strongly recommended.
\pdfoutput=1
% In particular, the hyperref package requires pdfLaTeX in order to break URLs across lines.

\documentclass[11pt]{article}

% Remove the "review" option to generate the final version.
\usepackage[]{latex/acl}

% Standard package includes
\usepackage{times}
\usepackage{latexsym}

% For proper rendering and hyphenation of words containing Latin characters (including in bib files)
\usepackage[T1]{fontenc}
% For Vietnamese characters
% \usepackage[T5]{fontenc}
% See https://www.latex-project.org/help/documentation/encguide.pdf for other character sets

% This assumes your files are encoded as UTF8
\usepackage[utf8]{inputenc}

% This is not strictly necessary, and may be commented out,
% but it will improve the layout of the manuscript,
% and will typically save some space.
\usepackage{microtype}

% This is also not strictly necessary, and may be commented out.
% However, it will improve the aesthetics of text in
% the typewriter font.
\usepackage{inconsolata}

% If the title and author information does not fit in the area allocated, uncomment the following
%
%\setlength\titlebox{<dim>}
%
% and set <dim> to something 5cm or larger.

\title{Avoiding Perjury of Large Language Models using Retrieval Augmentation}

% Author information can be set in various styles:
% For several authors from the same institution:
% \author{Author 1 \and ... \and Author n \\
%         Address line \\ ... \\ Address line}
% if the names do not fit well on one line use
%         Author 1 \\ {\bf Author 2} \\ ... \\ {\bf Author n} \\
% For authors from different institutions:
% \author{Author 1 \\ Address line \\  ... \\ Address line
%         \And  ... \And
%         Author n \\ Address line \\ ... \\ Address line}
% To start a separate ``row'' of authors use \AND, as in
% \author{Author 1 \\ Address line \\  ... \\ Address line
%         \AND
%         Author 2 \\ Address line \\ ... \\ Address line \And
%         Author 3 \\ Address line \\ ... \\ Address line}

\author{Tomás Vergara Browne \\
  Pontificia Universidad Católica de Chile \\
  \texttt{tomvergara@uc.cl} \\}

\begin{document}
\maketitle
\begin{abstract}
  Retrieval Augmentation provides the abilities to Large Language Models to access external information for them to generate more factual answers. Although the use of Retrieval Augmentation techniques has been intensely studied in the last year, a particular domain of legal domains has not been significantly studied, in comparison to other domains. In this paper, we explore the use of Retrieval Augmentation for the Mistral 7B model, which has surprised the open source community for its ability to supersede in performance models which are considerably larger. Through the use of the open source framework of RETA-LLM, we are able to add retrieval capabilities to the model. With this model, we are able to provide a new state of the art for the benchmark of LexGLUE.
\end{abstract}

\section{Introduction}


\section{Related Work}


\section{Methods}


\section{Results}

\section{Discussion}


\section{Appendices}


% Entries for the entire Anthology, followed by custom entries
\bibliography{anthology,custom}

\appendix


\end{document}
